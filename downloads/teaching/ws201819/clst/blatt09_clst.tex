\documentclass{julie-exercises}
\usepackage[utf8]{inputenc}
\usepackage{times}
\usepackage{graphics}
\usepackage{epsfig}
\usepackage{comment}
\usepackage{url}
\usepackage{verbatim}
\usepackage{listings}
\usepackage[ngerman]{babel}
\usepackage{t1enc}
\usepackage[utf8]{inputenc}
\usepackage{times}
\usepackage{graphics}
\usepackage{epsfig}
\usepackage{comment}
\usepackage{amsmath}
\usepackage{amsfonts}
\usepackage{amssymb}

\excludecomment{loesung}
%\includecomment{loesung}

\begin{document}
\course{Übung zur Vorlesung ``Einführung in die Computerlinguistik und Sprachtechnologie''}
\exno{9}
\semester{Wintersemester 2018/2019}
\duedate{30.01.2018, 23.59 Uhr; per
Email (PDF-Format) an
luise.modersohn@uni-jena.de}
\author{Prof. Dr. Udo Hahn, Luise Modersohn}
\date{24.01.2019}
\maketitle

\begin{aufgabe}{Potenzschreibung}{1}
	Schreiben Sie die Sprache $L_3$ in Potenzschreibung, orientieren Sie sich dafür am folgenden Beispiel: \\
	$ L_1 = \{ dabt, daabbt, daaabbbt, \ldots \} = d a^n b^n t (n \geq 1)$
	\vspace*{-8pt}

	\begin{itemize}
		\item $ L_3 = \{ aabbbc, aaaabbbbbc, aaaaaabbbbbbbc, \ldots \} $
	\end{itemize}
	\begin{loesung}
		{\bf L\"{o}sung:}
		$ = a^{2n} b^{2n+1} c (n \geq 1)$
	\end{loesung}
\end{aufgabe}


\begin{aufgabe}{Parsing}{9}
	\begin{teilaufgabe}{}{6}
		Beantworten Sie folgende Fragen:
		\begin{enumerate}
			\item Welche Schwierigkeiten ergeben sich beim Top-Down Parsing mit Tiefensuche durch syntaktische Ambiguität?
				\begin{loesung}

				\textbf{L\"osung:}
				Gefahr konkurrierende Interpretationen nicht zu finden, evtl. Backtracking n\"otig
				\end{loesung}

			\item Kann Top-Down Parsing mit rekursiven Regeln kombiniert werden?
				\begin{loesung}

				\textbf{L\"osung:}
				Nur mit rechtsrekursiven, sonst Endlosrekursion
				\end{loesung}

			\item Was ist mit der ``Left-Corner'' gemeint, wozu nutzen wir sie beim ``Left-Corner Parsing?''
				\begin{loesung}

				\textbf{L\"osung:}
				Die ``Left-Corner'' ist das erste (genauer das am weitesten links stehende) Non-Terminal-Symbol auf der rechten Regelseite: NP $\rightarrow$ \textbf{DET} N\\
				Beim ``Left-Corner Parsing'' dient diese dem Abgleich mit der Eingabe in der Top-Down-Prediction-Phase, also der begrenzten Vorhersage.
				\end{loesung}
		\end{enumerate}
	\end{teilaufgabe}


	\begin{teilaufgabe}{}{6}
		In dieser Aufgabe sollen Sie drei unterschiedliche Parsingverfahren am Satz \textit{Fly to Detroit} vorführen. Verwenden Sie dafür die folgende Grammatik:\\
		\begin{tabbing}
		G = \=( N, T, P, S ) mit \\
		N = \> \{ NP, VP, PP, det, n, v, p \} \\
		T = \> \{ fly, to, Detroit\} \\
		P = \> \{	\= S $\rightarrow$ VP PP, \\
			\>		\> NP $\rightarrow$ n, \\
			\>		\> VP $\rightarrow$ v PP, \\
			\>		\> VP $\rightarrow$ v, \\
			\>		\> PP $\rightarrow$ p NP, \\
			\>		\> n $\rightarrow$ Detroit, \\
			\>		\> p $\rightarrow$ to, \\
			\>		\> v $\rightarrow$ Fly \}\\
		\end{tabbing}

		%\begin{teilaufgabe}{\small{(1 Punkt)}}{}
		%Erklären Sie zu Beginn kurz was mit ``Left-Corner'' gemeint ist und wozu wir sie beim ``Left-Corner Parsing'' nutzen?
		%\begin{loesung}
		%\\\textbf{L"{o}sung}:\\
		%Die ``Left-Corner'' ist das erste (genauer das am weitesten links stehende) Non-Terminal-Symbol auf der rechten Regelseite: NP $\rightarrow$ \textbf{DET} N\\
		%Beim ``Left-Corner Parsing'' dient diese dem Abgleich mit der Eingabe in der Top-Down-Prediction-Phase.
		%\end{loesung}
		%\end{teilaufgabe}{}{}


		Führen Sie einen \textbf{Top-Down Parse} (mit Tiefensuche), einen \textbf{Bottom-Up Parse} und einen \textbf{Left-Corner Parse} für den Beispielsatz durch. Zeichnen Sie dazu für jeden Schritt den (Teil-)Parsebaum. Führen Sie für die ambigen Regeln (VP-Ableitungen) beide m\"oglichen Analysen durch -- abhängig vom Verfahren kann dies Backtracking n\"otig machen.
		% \begin{loesung}
		% 	\begin{figure}[h]
		%  		\centering
		% 	 	\includegraphics[scale=0.85]{fig/td-parse.jpg}
		% 	 	\caption{Top-Down}
		% 	\end{figure}
		% 	\begin{figure}[h]
		%  		\centering
		% 	 	\includegraphics[scale=0.85]{fig/bu-parse.jpg}
		% 	 	\caption{Bottom-Up}
		% 	\end{figure}
		% 		\begin{figure}[h]
		%  		\centering
		% 	 	\includegraphics[scale=0.85]{fig/lc-parse.jpg}
		% 	 	\caption{Left-Corner}
		% 	\end{figure}
		% \end{loesung}

		\begin{loesung}
			\begin{figure}[h]
		 		\centering
			 	\includegraphics[width = \textwidth]{Fly-to-Detroit-TD.pdf}
			 	%\caption{Top-Down}
			\end{figure}
			\begin{figure}[h]
		 		\centering
			 	\includegraphics[width = \textwidth]{Fly-to-Detroit-BU.pdf}
			 	%\caption{Bottom-Up}
			\end{figure}
				\begin{figure}[h]
		 		\centering
			 	\includegraphics[width = \textwidth]{Fly-to-Detroit-LC.pdf}
			 	%\caption{Left-Corner}
			\end{figure}
		\end{loesung}

	\end{teilaufgabe}
\end{aufgabe}

\end{document}
